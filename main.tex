\documentclass{article}
% Begin Preamble
\usepackage[utf8]{inputenc}
\usepackage{natbib}
\usepackage{graphicx}
\usepackage{amsfonts}
\usepackage{fullpage}
\usepackage[english]{babel}
\usepackage{color}
\usepackage{amsmath}
\usepackage{url}
\usepackage{standalone}
\usepackage{parskip}
\usepackage{graphicx}
\usepackage{caption}
\usepackage{subcaption}

    %\graphicspath{ {./Figures/} }

\title{Lecture 5: Operators and the Schr\"{o}dinger Equation}
\author{Lauren Shriver}
\date{October 28 2018}
% End Preamble
\begin{document}

\maketitle

\section*{Postulates Discussed So Far}
\begin{enumerate}
    \item Particles/systems are governed/described by \textbf{wave functions} $\Psi(x)$
        \begin{itemize}
            \item Each $\Psi(x)$ must be a complex number, continuous, and normalizable
            \item Note: the derivative of the wave function with respect to $x$ need not be continuous 
        \end{itemize}
    \item Associated with any $\Psi(x)$ is a probability $\mathbb{P}(x)dx$
        \begin{itemize}
            \item Note: since $x$ is a continuous variable, we write $\mathbb{P}(x)dx$ (rather than just $\mathbb{P}(x)$) to dentoe the probability our system/particle is located somewhere in the range $(x,x+dx)$
            \item Recall: for continuous probability distributions, the probability of our system having a discrete value is always zero
            \item Equation for this relation: 
                \begin{equation}
                    \mathbb{P}(x)dx=|\Psi(x)|^2dx
                \end{equation}
        \end{itemize}
    \item A wave function can be a superposition of allowed states 
        \begin{itemize}
            \item Example: 
                \begin{equation}
                    \Psi(x) = \alpha\Psi_1(x) + \beta\Psi_2(x) 
                \end{equation}
        \end{itemize}
\end{enumerate}
\section*{Postulate 4}
For each \textbf{observable} $a$, we have an associated operator $\hat{A}$.
\begin{itemize}
    \item Examples so far:
        \begin{itemize}
            \item Momentum: $\hat{p}\equiv\frac{\hbar}{i}\frac{\partial}{\partial x}$
            \item Position: $\hat{x}\equiv x$
            \item Energy: $\hat{E}\equiv -\frac{\hbar^2}{2m}\frac{\partial^2}{\partial x^2} + V(x)$
        \end{itemize}
\end{itemize}
\section*{Postulate 5}
Upon measuring an observable $a$ associated with an operator $\hat{A}$, two things will occur:
\begin{enumerate}
    \item The measured value will be one of the eigenvalues of $\hat{A}$
    \item After measurement, the particle/system collapses into an eigenfunction $\Psi_a$
\end{enumerate}
\section*{Postulate 6}
Given an operator $\hat{A}$, we can normalize its set of eignefunctions by assuming they are all orthonormal to one another. This is summarized by the following normalization condition:
\begin{equation}
    \int{\psi_a^*(x) \psi_b(x) dx} = \delta_{ab}
\end{equation}
\section*{Schr\"{o}dinger Equation}
\begin{equation}
    \begin{split}
        i\hbar \frac{\partial}{\partial t} \Psi(x,t) & = \hat{E}\Psi(x,t) 
        \\ 
        \\
        & = -\frac{\hbar^2}{2m} \frac{\partial^2}{\partial x^2} \Psi(x,t) + V(x)\Psi(x,t) 
    \end{split}
\end{equation}
\end{document}
